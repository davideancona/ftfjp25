\section{Runtime verification of use-site object protocols}\label{use-site}

The concept of typestate \cite{StromYemini86,GarciaTWA14} has been introduced as a refinement of the classical notion of type to specify and verify valid sequences of operations on objects.

Typestates are behavioral types \cite{AnconaBB0CDGGGH16}, as global \cite{CarboneHY07} and session types \cite{Honda93}, with the difference that the former have been initially devised mainly in the context of sequential programming, while the latter have been introduced to specify and verify bidirectional communication protocols in concurrent programs. Indeed, the aim of typestates is the definition and checking of correct \emph{object protocols} \cite{BacchianiBGMR22}.

\subsection{Object protocols for single interfaces}\label{iterator}
The iterator design pattern in OOP offers a typical and intuitive example of object protocol which can be defined with a typestate. Indeed, while in Java the generic type \lstinline|Iterator<T>| of package \lstinline|java.util| only defines the allowed operations on an iterator object, one may be interested in verifying that the following protocol is respected when iterators are used:
\begin{enumerate}
 \item method \lstinline|it.next()| can be safely called only if a previous method has been called on \lstinline|it|, such method was \lstinline|hasNext()|, and its returned value was true;
 \item optionally, consecutive calls to method \lstinline|hasNext()| on the same object should be avoided.
\end{enumerate}

An important aspect of the requirements above is that they define the \emph{use-site} protocol for iterators. Under this point of view, a definition  for such requirements, expressed in any suitable formalism, should be associated with the \lstinline|Iterator|\footnote{Hence, to all classes implementing it.} interface to refine it.

To verify an object protocol several solutions have been proposed in literature.
Most of the static approaches rely on typestates \cite{KouzapasDPG18,VoineaDG20,MotaGR21,BacchianiBGMR22}, with several limitations in presence of concurrency, inheritance and aliasing.

In some programming languages the implementation and runtime verification of protocols are hardcoded in the code of the objects. For instance, this policy is adopted in Java, not only with iterators, but also with other objects like regular expression matchers in \lstinline|java.util.regex| or streams in \lstinline|java.util.stream|. In these cases, sequences of method calls not compliant with the use-site protocol throw \lstinline|IllegalStateException|.
Although this approach provides cast-iron guarantees and simplifies testing and debugging, it also exhibits drawbacks: implementation of objects is more  complex and less efficient. Furthermore, checking of the protocol cannot be disabled, and the code for verifying it cannot be easily shared and reused.

Classical tools and frameworks for unit testing based on dynamic assertion checking \cite{LangrHT15,FardM17} are not suited for this aim, since they have been devised to test the correctness of classes and objects against the specification of the implemented interface(s), that is, to verify definition-site object protocols \cref{definition-site}. Unit testing is used to dynamically check that a sequence of calls to methods of an interface behaves as expected, but cannot verify whether some client code calls sequences of such methods in a correct way.

The approach followed here is inspired by RV and is based on the following assumptions:
\begin{itemize}
 \item The tested code is automatically instrumented to generate the relevant events for assertion checking.

       The following six types of events are typically needed: (1) method, constructor called or returned, (2) method, constructor entered or exited, (3) exception thrown by a method or constructor, (4) method or constructor exited with a thrown examples.

       All events are equipped with the expected information, like the name of the method, the arguments, and the returned value. Though redundant in some cases, (1)+(2), and (3)+(4) allow tracing of callbacks and calls of library methods or constructors without any need to instrument the imported libraries.
 \item A textual formalism to specify the object protocol associated with one or more interfaces. From the specification a monitor should be automatically generated to recognize the correct sequences of events. For the examples in this paper \rml is used, but any other formalism which is expressive enough would work as well.
 \item An annotation mechanism for associating the code under testing with the specifications of the corresponding protocols. Moreover, in statically typed languages with nominal types, specifications should be associated with the types (preferably interfaces) of the objects that have to be checked.
\end{itemize}

The specification for iterators can be provided in an \rml file:
\begin{lstlisting}[basicstyle=\tt\footnotesize]
// file "iterator.rml"      
Main={let id; newIter(id)(Iterator<id> | Main)};
Iterator<id>=((hasNext(id,true)+ next(id))* hasNext(id,false)+);   
Optional={let id; newIter(id)(NoRep<id> | Optional)};
NoRep<id>=(hasNext(id) noHasNext(id))*; 
\end{lstlisting}
The specification \lstinline|Optional| adds a further check to avoid redundant calls to \lstinline|hasNext()|. Any call of shape \lstinline|id.hasNext()| must be followed by an event of different type: either a call to \lstinline|next()| on any object, or a call to \lstinline|hasNext()| to an object different from \lstinline|id|.

A possible setting for testing the protocol inside the body of a method would be:
\begin{lstlisting}[basicstyle=\tt\footnotesize]
@CheckProtocol(fromFile="iterator.rml", 
      name="Main", types=Iterator.class)
@CheckProtocol(fromFile="iterator.rml", 
      name="Optional", types=Iterator.class)
public void m(){ // method under test
      // uses iterators
}
\end{lstlisting}
In the example, all calls of methods with targets of type \lstinline|Iterator| will be checked against the specifications defined by \lstinline|Main| and \lstinline|Optional| in file \lstinline|iterator.rml|.

The body of \lstinline|m()| is instrumented to generate all relevant events, corresponding to method calls, which are sent to the monitors which are compiled from the specifications in \lstinline|iterator.rml|. Communication between the code under test and monitors
can be implemented with sockets.


